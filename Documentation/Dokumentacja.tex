\documentclass{report}

\usepackage[T1]{fontenc}
\usepackage[polish]{babel}
\usepackage[utf8]{inputenc}
\usepackage{lmodern}
\usepackage{longtable}
\usepackage{multirow}
\usepackage{enumerate}
\usepackage{array}
\newcolumntype{C}[1]{>{\centering\let\newline}}
\selectlanguage{polish}


\title{Dokumentacja projektu "Booka"}
\author{Miłosz Białczak (218295)\\ Mateusz Gniewkowski (218138)\\ Beata Szeląg (218139)}
\date{\today}

\begin{document}

\setlength{\LTleft}{-20cm plus -1fill}
\setlength{\LTright}{\LTleft}
\maketitle
\tableofcontents{}



\chapter{Wstęp}
\chapter{Analiza systemu}

\section{Wymagania funkcjonalne}


%\begin{longtable}{|c|p{12cm}|}
%\caption{Wymaganie funkcjonalne F\_XX} \label{tab:F_XX} \\ \hline
%\multicolumn{2}{ |c| }{Nazwa wymagania} \\ \hline
%ID & F\_XX \\ \hline
%Opis & 	opis \\ \hline
%Priorytet & wymagane/oczekiwane/opcjonalne \\ \hline
%Powiązania & login/-  \\ \hline
%\end{longtable} 




\begin{longtable}{|c|p{12cm}|}
\caption{Wymaganie funkcjonalne F\_00} \label{tab:F_00} \\ \hline
\multicolumn{2}{ |c| }{Utworzenie konta} \\ \hline
ID & F\_00 \\ \hline
Opis & Użytkownicy nieposiadający kont mają możliwość ich założenia. Po wybraniu odpowiedniej opcji, na podany adres mailowy zostaje wysłana wiadomość zawierająca link potwierdzający rejestrację.  \\ \hline
Priorytet & wymagane\\ \hline
Powiązania & - \\ \hline
\end{longtable} 


\begin{longtable}{|c|p{12cm}|}
\caption{Wymaganie funkcjonalne F\_01} \label{tab:F_01} \\ \hline
\multicolumn{2}{ |c| }{Logowanie} \\ \hline
ID & F\_01 \\ \hline
Opis & 	Użytkownik posiadający konto może zalogować się do systemu\\ \hline
Priorytet & wymagane\\ \hline
Powiązania & - \\ \hline
\end{longtable} 


\begin{longtable}{|c|p{12cm}|}
\caption{Wymaganie funkcjonalne F\_02} \label{tab:F_02} \\ \hline
\multicolumn{2}{ |c| }{Wylogowanie} \\ \hline
ID & F\_02 \\ \hline
Opis & Zalogowany użytkownik może się wylogować \\ \hline
Priorytet & wymagane\\ \hline
Powiązania & - \\ \hline
\end{longtable}

\begin{longtable}{|c|p{12cm}|}
\caption{Wymaganie funkcjonalne F\_03} \label{tab:F_03} \\ \hline
\multicolumn{2}{ |c| }{Konfiguracja} \\ \hline
ID & F\_03 \\ \hline
Opis & 	Użytkownik ma możliwość konfiguracji swojego konta, a w szczególności ustawienia i zmiany hasła \\ \hline
Priorytet & wymagane\\ \hline
Powiązania & - \\ \hline
\end{longtable} 



\begin{longtable}{|c|p{12cm}|}
\caption{Wymaganie funkcjonalne F\_04} \label{tab:F_04} \\ \hline
\multicolumn{2}{ |c| }{Dodawanie pozycji do księgozbioru} \\ \hline
ID & F\_04 \\ \hline
Opis & 	Użytkownik może dodać nową pozycję do swojego księgozbioru. Powinien on określić jej tytuł, autora i kategorię. Jeżeli użytkownik posiada elektroniczną kopię książki na dysku lokalnym, lub obsługiwanym dysku sieciowym (serwer FPT), może ją zaimportować automatycznie. Jest możliwe importowanie całych katalogów naraz - aplikacja powinna reagować na zmiany w danym katalogu i automatycznie importować ich zawartość. \\ \hline
Priorytet & wymagane \\ \hline
Powiązania & -  \\ \hline
\end{longtable} 

\begin{longtable}{|c|p{12cm}|}
\caption{Wymaganie funkcjonalne F\_05} \label{tab:F_05} \\ \hline
\multicolumn{2}{ |c| }{Usuwanie pozycji z księgozbioru} \\ \hline
ID & F\_05 \\ \hline
Opis & 	Użytkownik może usunąć dowolną pozycję ze swojego księgozbioru. \\ \hline
Priorytet & wymagane\\ \hline
Powiązania & -  \\ \hline
\end{longtable}

\begin{longtable}{|c|p{12cm}|}
\caption{Wymaganie funkcjonalne F\_06} \label{tab:F_06} \\ \hline
\multicolumn{2}{ |c| }{Podgląd księgozbioru} \\ \hline
ID & F\_06 \\ \hline
Opis & 	Użytkownik może przeglądać swój księgozbiór. \\ \hline
Priorytet & wymagane \\ \hline
Powiązania & -  \\ \hline
\end{longtable} 

\begin{longtable}{|c|p{12cm}|}
\caption{Wymaganie funkcjonalne F\_07} \label{tab:F_07} \\ \hline
\multicolumn{2}{ |c| }{Przeszukiwanie księgozbioru} \\ \hline
ID & F\_07 \\ \hline
Opis & Użytkownik może przeszukiwać swój księgozbiór po nazwie, autorze, kategorii itp.\\ \hline
Priorytet & wymagane \\ \hline
Powiązania & F\_06  \\ \hline
\end{longtable} 

\begin{longtable}{|c|p{12cm}|}
\caption{Wymaganie funkcjonalne F\_08} \label{tab:F_08} \\ \hline
\multicolumn{2}{ |c| }{Przeszukiwanie zasobów bibliotecznych} \\ \hline
ID & F\_08 \\ \hline
Opis & Użytkownik powinien mieć możliwość przeglądania zasobów bibliotek miejskich. \\ \hline
Priorytet & wymagane \\ \hline
Powiązania & -  \\ \hline
\end{longtable}

\begin{longtable}{|c|p{12cm}|}
\caption{Wymaganie funkcjonalne F\_09} \label{tab:F_09} \\ \hline
\multicolumn{2}{ |c| }{Przeszukiwanie sklepów internetowych} \\ \hline
ID & F\_09 \\ \hline
Opis & Użytkownik powinien mieć możliwość wyszukiwania książek w zdefiniowanych sklepach \\ \hline
Priorytet & oczekiwane  \\ \hline
Powiązania & -  \\ \hline
\end{longtable}

\begin{longtable}{|c|p{12cm}|}
\caption{Wymaganie funkcjonalne F\_10} \label{tab:F_10} \\ \hline
\multicolumn{2}{ |c| }{Pożyczanie książek} \\ \hline
ID & F\_10 \\ \hline
Opis & Użytkownik powinien mieć możliwość oznaczenia książki jako pożyczonej (z uwzględnieniem komu owa książka została pożyczona). Informacja o pożyczonych komuś i od kogoś książkach powinna być dostępna w widoku księgozbioru.\\ \hline
Priorytet & wymagane \\ \hline
Powiązania & F\_06, F\_11   \\ \hline
\end{longtable} 

\begin{longtable}{|c|p{12cm}|}
\caption{Wymaganie funkcjonalne F\_11} \label{tab:F_11} \\ \hline
\multicolumn{2}{ |c| }{Oddawanie książek} \\ \hline
ID & F\_11 \\ \hline
Opis & System powinien dawać możliwość oddawania książek. \\ \hline
Priorytet & wymagane \\ \hline
Powiązania & F\_06, F\_10  \\ \hline
\end{longtable}

\begin{longtable}{|c|p{12cm}|}
\caption{Wymaganie funkcjonalne F\_12} \label{tab:F_12} \\ \hline
\multicolumn{2}{ |c| }{System notyfikacji} \\ \hline
ID & F\_12 \\ \hline
Opis & System powinien automatycznie informować użytkownika o akcjach z nim związanych (np. informacja o zbliżającym się terminie oddania książki). Notyfikacje powinny być możliwe do modyfikacji i wyłączenia.  \\ \hline
Priorytet & oczekiwane \\ \hline
Powiązania & -  \\ \hline
\end{longtable}

\begin{longtable}{|c|p{12cm}|}
\caption{Wymaganie funkcjonalne F\_13} \label{tab:F_13} \\ \hline
\multicolumn{2}{ |c| }{Dostęp do pozycji w wersjach elektronicznych} \\ \hline
ID & F\_13 \\ \hline
Opis & Jeżeli książka jest dostępna w wersji elektronicznej, powinno być możliwe otworzenie jej z poziomu aplikacji. \\ \hline
Priorytet & oczekiwane \\ \hline
Powiązania & -  \\ \hline
\end{longtable} 


\begin{longtable}{|c|p{12cm}|}
\caption{Wymaganie funkcjonalne F\_14} \label{tab:F_14} \\ \hline
\multicolumn{2}{ |c| }{Udostępnianie księgozbioru} \\ \hline
ID & F\_14 \\ \hline
Opis & Użytkownik powinien mieć możliwość udostępnienia swojego księgozbioru do podglądu innym użytkownikom (niekoniecznie z opcją czytania książek w wersji elektronicznej) \\ \hline
Priorytet & opcjonalne \\ \hline
Powiązania & -  \\ \hline
\end{longtable} 

\begin{longtable}{|c|p{12cm}|}
\caption{Wymaganie funkcjonalne F\_15} \label{tab:F_15} \\ \hline
\multicolumn{2}{ |c| }{Wysyłanie powiadomień do innych użytkowników} \\ \hline
ID & F\_15 \\ \hline
Opis & Użytkownik powinien mieć możliwość wysyłania powiadomień innym użytkownikom. Powiadomienia mogą dotyczyć próśb o pożyczenie, oddanie książki, udostępnienie podglądu księgozbioru itp. Powiadomienia mogą być wysyłane poprzez Facebooka, e-maila, lub aplikację. \\ \hline
Priorytet & oczekiwane \\ \hline
Powiązania & -  \\ \hline
\end{longtable}



\section{Wymagania niefunkcjonalne}

%\begin{longtable}{|c|p{12cm}|}
%\caption{Wymaganie niefunkcjonalne N\_XX} \label{tab:N_XX} \\ \hline
%\multicolumn{2}{ |c| }{Nazwa wymagania} \\ \hline
%ID & N\_XX \\ \hline
%Opis & opis \\ \hline
%Priorytet & wymagane/oczekiwane/opcjonalne \\ \hline
%Powiązania & login/- \\ \hline
%\end{longtable}

\begin{longtable}{|c|p{12cm}|}
\caption{Wymaganie niefunkcjonalne N\_00} \label{tab:N_00} \\ \hline
\multicolumn{2}{ |c| }{Interfejs użytkownika} \\ \hline
ID & N\_00 \\ \hline
Opis & Aplikacja powinna posiadać ładny i przejrzysty graficzny interfejs użytkownika, możliwe intuicyjny i łatwy w obsłudze. \\ \hline
Priorytet & oczekiwane \\ \hline
Powiązania & - \\ \hline
\end{longtable} 


\begin{longtable}{|c|p{12cm}|}
\caption{Wymaganie niefunkcjonalne N\_01} \label{tab:N_01} \\ \hline
\multicolumn{2}{ |c| }{Odporność na utratę danych} \\ \hline
ID & N\_01 \\ \hline
Opis & Stworzenie mechanizmów w systemie odpowiedzialnych za cykliczną
archiwizację stanu bazy danych oraz możliwość wczytania jej z pliku. \\ \hline
Priorytet & oczekiwane \\ \hline
Powiązania & - \\ \hline
\end{longtable}

\begin{longtable}{|c|p{12cm}|}
\caption{Wymaganie niefunkcjonalne N\_02} \label{tab:N_02} \\ \hline
\multicolumn{2}{ |c| }{Skalowalność} \\ \hline
ID & N\_02 \\ \hline
Opis & System powinien zapewnić możliwość jego rozbudowy pod względem rozmiaru \\ \hline
Priorytet & oczekiwane \\ \hline
Powiązania & - \\ \hline
\end{longtable}

\begin{longtable}{|c|p{12cm}|}
\caption{Wymaganie niefunkcjonalne N\_03} \label{tab:N_03} \\ \hline
\multicolumn{2}{ |c| }{Bezpieczeństwo danych} \\ \hline
ID & N\_03 \\ \hline
Opis & System powinien dbać o zapewnienie ograniczonego
dostępu do przechowywanych informacji. \\ \hline
Priorytet & wymagane \\ \hline
Powiązania & - \\ \hline
\end{longtable}

\begin{longtable}{|c|p{12cm}|}
\caption{Wymaganie niefunkcjonalne N\_04} \label{tab:N_04} \\ \hline
\multicolumn{2}{ |c| }{Wydajność} \\ \hline
ID & N\_04 \\ \hline
Opis & System powinien reagować na zapytania użytkownika z opóźnieniem nie większym niż pół sekundy. \\ \hline
Priorytet & oczekiwane \\ \hline
Powiązania & - \\ \hline
\end{longtable}


\begin{longtable}{|c|p{12cm}|}
\caption{Wymaganie niefunkcjonalne N\_05} \label{tab:N_05} \\ \hline
\multicolumn{2}{ |c| }{Możliwość rozwoju} \\ \hline
ID & N\_05 \\ \hline
Opis & System powinien być zaplanowany w ten sposób, aby umożliwić dołączanie nowych funkcjonalności. \\ \hline
Priorytet & oczekiwane \\ \hline
Powiązania & - \\ \hline
\end{longtable}


\section{Wymagania technologiczne}


%\begin{longtable}{|c|p{12cm}|}
%\caption{Wymaganie technologiczne T\_XX} \label{tab:T_XX} \\ \hline
%\multicolumn{2}{ |c| }{Nazwa wymagania} \\ \hline
%ID & T\_XX \\ \hline
%Opis & opis \\ \hline
%Priorytet & wymagane/oczekiwane/opcjonalne \\ \hline
%Powiązania & login/- \\ \hline
%\end{longtable} 



\begin{longtable}{|c|p{12cm}|}
\caption{Wymaganie technologiczne T\_00} \label{tab:T_00} \\ \hline
\multicolumn{2}{ |c| }{Java} \\ \hline
ID & T\_00 \\ \hline
Opis & Aplikacja powinna być stworzona w języku Java, wersja 8 lub wyższa. \\ \hline
Priorytet & wymagane \\ \hline
Powiązania & - \\ \hline
\end{longtable} 


\begin{longtable}{|c|p{12cm}|}
\caption{Wymaganie technologiczne T\_01} \label{tab:T_01} \\ \hline
\multicolumn{2}{ |c| }{SpringBoot} \\ \hline
ID & T\_01 \\ \hline
Opis & Serwer aplikacji powinien być napisany z wykorzystaniem framework'a Spring Boot. \\ \hline
Priorytet & wymagane\\ \hline
Powiązania & - \\ \hline
\end{longtable} 


\begin{longtable}{|c|p{12cm}|}
\caption{Wymaganie technologiczne T\_02} \label{tab:T_02} \\ \hline
\multicolumn{2}{ |c| }{MySQL} \\ \hline
ID & T\_02 \\ \hline
Opis & Wykorzystywanym systemem bazodanowym powinien być MySQL. \\ \hline
Priorytet & wymagane \\ \hline
Powiązania & - \\ \hline
\end{longtable}


\begin{longtable}{|c|p{12cm}|}
\caption{Wymaganie technologiczne T\_03} \label{tab:T_03} \\ \hline
\multicolumn{2}{ |c| }{Aplikacja webowa} \\ \hline
ID & T\_03 \\ \hline
Opis & Aplikacja webowa powinna być napisana z wykorzystaniem technologii CSS, HTML i JavaScript z framework'iem AngularJS  \\ \hline
Priorytet & wymagane \\ \hline
Powiązania & - \\ \hline
\end{longtable} 

\section{Przypadki użycia}

Użyte skróty:
\begin{itemize}
\item WP - warunki początkowe
\item WK - warunki końcowe
\end{itemize}


\begin{longtable}{|c|p{12cm}|}
\caption{Przypadek użycia PU\_00} \label{tab:PU_00} \\ \hline
\multicolumn{2}{ |c| }{Zakładanie konta} \\ \hline
ID & PU\_00 \\ \hline
Cel & Możliwość stworzenia nowego konta do systemu \\ \hline
WP & - \\ \hline
WK & Poprawne zarejestrowanie nowego użytkownika \\ \hline
\multirow{4}{*}{Przebieg} 
& 1. Należy wybrać opcję 'Sign up'. \\
& 2. Należy podać imię, nazwisko, email, hasło oraz opcjonalnie wypełnić dodatkowe pola \\
& 3. Należy zatwierdzić formularz \\
& 4. Na email zostanie przesłany link potwierdzający rejestrację, który należy kliknąć \\
\hline
\end{longtable} 

\begin{longtable}{|c|p{12cm}|}
\caption{Przypadek użycia PU\_01} \label{tab:PU_01} \\ \hline
\multicolumn{2}{ |c| }{Logowanie} \\ \hline
ID & PU\_01 \\ \hline
Cel & Możliwość zalogowania się do systemu \\ \hline
WP & Poprawnie utworzone konto użytkownika \\ \hline
WK & Zalogowanie się do systemu \\ \hline
\multirow{3}{*}{Przebieg} 
& 1. Należy wybrać opcję 'Sign in'. \\
& 2. Należy podać email oraz hasło \\
& 3. Należy zatwierdzić dane \\
\hline
\end{longtable} 
\begin{longtable}{|c|p{12cm}|}
\caption{Przypadek użycia PU\_02} \label{tab:PU_02} \\ \hline
\multicolumn{2}{ |c| }{Edycja danych użytkownika} \\ \hline
ID & PU\_02 \\ \hline
Cel & Możliwość edycji danych użytkownika \\ \hline
WP & Poprawnie zalogowanie się do systemu \\ \hline
WK & Zmiana danych użytkownika na nowe \\ \hline
\multirow{3}{*}{Przebieg} 
& 1. Należy wybrać opcję 'Settings' oraz 'Profile'. \\
& 2. Należy wybrać interesujące nas dane, które chcemy zmienić i wybrać opcję 'Edit' \\
& 3. Po wprowadzeniu danych należy zatwierdzić zmiany \\
\hline
\end{longtable} 

\begin{longtable}{|c|p{12cm}|}
\caption{Przypadek użycia PU\_03} \label{tab:PU_03} \\ \hline
\multicolumn{2}{ |c| }{Przeglądanie księgozbioru} \\ \hline
ID & PU\_03 \\ \hline
Cel & Możliwość przeglądania wszystkich pozycji w księgozbiorze \\ \hline
WP & Poprawnie zalogowanie się \\ \hline
WK & Wyświetlenie listy książek \\ \hline
\multirow{1}{*}{Przebieg} 
& 1. Należy wybrać opcję 'Books' \\
\hline
\end{longtable} 

\begin{longtable}{|c|p{12cm}|}
\caption{Przypadek użycia PU\_04} \label{tab:PU_04} \\ \hline
\multicolumn{2}{ |c| }{Wyszukiwanie książek z własnego księgozbioru} \\ \hline
ID & PU\_04 \\ \hline
Cel & Możliwość przeglądania wybranych pozycji w księgozbiorze \\ \hline
WP & Poprawnie zalogowanie się \\ \hline
WK & Wyświetlenie listy książek \\ \hline
\multirow{2}{*}{Przebieg} 
& 1. Należy wybrać opcję 'Books' \\
& 2. Należy wypełnić jeden/kilka filtrów: filtr imion autorów ('Name'), nazwisk autorów ('Surname'), tytułów ('Title'), tagów ('Tag') \\
\hline
\end{longtable} 

\begin{longtable}{|c|p{12cm}|}
\caption{Przypadek użycia PU\_05} \label{tab:PU_05} \\ \hline
\multicolumn{2}{ |c| }{Dodanie nowej książki do księgozbioru} \\ \hline
ID & PU\_05 \\ \hline
Cel & Możliwość dodania nowej pozycji do księgozbioru \\ \hline
WP & Poprawnie zalogowanie się \\ \hline
WK & Dodanie nowej pozycji do księgozbioru \\ \hline
\multirow{3}{*}{Przebieg} 
& 1. Należy wybrać opcję 'Books' i 'Add book' \\
& 2. Należy uzupełnić formularz, podając imię i nazwisko autora, tytuł książki i opcjonalnie wypełnić pozostałe pola \\
& 3. Należy zatwierdzić formularz \\
\hline
\end{longtable} 


\begin{longtable}{|c|p{12cm}|}
\caption{Przypadek użycia PU\_06} \label{tab:PU_06} \\ \hline
\multicolumn{2}{ |c| }{Edycja danych książki} \\ \hline
ID & PU\_06 \\ \hline
Cel & Możliwość edycji informacji o danej książce \\ \hline
WP & Poprawnie zalogowanie się \\ \hline
WK & Zmiana danych dotyczących danej książki \\ \hline
\multirow{4}{*}{Przebieg} 
& 1. Należy wybrać opcję 'Books' \\
& 2. Należy wybrać daną książkę poprzez 'See details' \\
& 3. Należy wybrać interesujące nas dane, które chcemy zmienić i wybrać opcję 'Edit' \\
& 4. Po wprowadzeniu danych należy zatwierdzić zmiany \\
\hline
\end{longtable}

\begin{longtable}{|c|p{12cm}|}
\caption{Przypadek użycia PU\_07} \label{tab:PU_07} \\ \hline
\multicolumn{2}{ |c| }{Usuwanie książki} \\ \hline
ID & PU\_07 \\ \hline
Cel & Możliwość usunięcia danej pozycji z księgozbioru \\ \hline
WP & Poprawnie zalogowanie się \\ \hline
WK & Usunięcie pozycji z księgozbioru \\ \hline
\multirow{4}{*}{Przebieg} 
& 1. Należy wybrać opcję 'Books' \\
& 2. Należy wybrać daną książkę poprzez 'See details' \\
& 3. Należy wybrać opcję 'Delete' \\
& 4. W dodatkowym oknie należy potwierdzić chęć usunięcia danej pozycji (opcja 'Yes') \\
\hline
\end{longtable}



\begin{longtable}{|c|p{12cm}|}
\caption{Przypadek użycia PU\_08} \label{tab:PU_08} \\ \hline
\multicolumn{2}{ |c| }{Podgląd książki elektronicznej} \\ \hline
ID & PU\_08 \\ \hline
Cel & Możliwość otworzenia książki elektronicznej \\ \hline
WP & Poprawnie zalogowanie się oraz posiadanie książki/książek elektronicznych w księgozbiorze\\ \hline
WK & Otworzenie książki w programie typu Adobe Reader / Foxit \\ \hline
\multirow{3}{*}{Przebieg} 
& 1. Należy wybrać opcję 'Books' \\
& 2. Należy wybrać daną pozycję \\
& 3. Należy wybrać ikonę książki \\
\hline
\end{longtable}
\break
\begin{longtable}{|c|p{12cm}|}
\caption{Przypadek użycia PU\_09} \label{tab:PU_09} \\ \hline
\multicolumn{2}{ |c| }{Pożyczenie książki z własnego księgozbioru} \\ \hline
ID & PU\_09 \\ \hline
Cel & Możliwość oznaczenia własnej książki jako pożyczonej wraz z oznaczeniem komu i kiedy została ona pożyczona \\ \hline
WP & Poprawnie zalogowanie się oraz posiadanie książki/książek w księgozbiorze\\ \hline
WK & Oznaczenie książki jako pożyczonej / Foxit \\ \hline
\multirow{5}{*}{Przebieg} 
& 1. Należy wybrać opcję 'Books' \\
& 2. Należy wybrać daną pozycję \\
& 3. Należy wybrać opcję 'See details' oraz 'Lend' \\
& 4. Należy wybrać komu pożyczamy książkę (podajemy login osoby, jeśli posiada ona konto w systemie, w przeciwnym wypadku - dowolną nazwę) oraz opcjonalnie uzupełniamy resztę informacji (np: określenie daty zwrotu).\\
& 5. Zatwierdzamy formularz \\
\hline
\end{longtable}


\chapter{Projekt systemu}

\section{REST API}


\begin{longtable}{|p{3cm}|p{2cm}|p{3.5cm}|p{6cm}|}
\caption{Akcje związane z użytkownikami} \label{API_0} \\ \hline
\multicolumn{4}{ |c| }{ Użytkownicy } \\ 
\multicolumn{4}{ |c| }{ /users } \\ \hline
Endpoint & Request & Opis & Dodatkowo \\ \hline
/users/ & GET & Pobranie informacji o użytkownikach & - \\ \hline
/users/<user\_id> & GET & Wyświetlanie informacji o użytkowniku & - \\ \hline
/users/<user\_id> & PUT & Zmiana danych użytkownika & body: login, password \\ \hline
/users/<user\_id> & DELETE & Usuwanie konta użytkownika & - \\ \hline
/users/sign\_in & POST & Logowanie & body: login, password \\ \hline
\end{longtable} 


\begin{longtable}{|p{4.0cm}|p{1.5cm}|p{3.5cm}|p{5.5cm}|}
\caption{Akcje związane z książkami} \label{API_1} \\ \hline
\multicolumn{4}{ |c| }{ Książki } \\ 
\multicolumn{4}{ |c| }{ /books } \\ \hline
Endpoint & Request & Opis & Dodatkowo \\ \hline
/books/ & GET & Pobranie informacji o książkach & - \\ \hline
/books/<book\_id> & GET & Wyświetlanie informacji o danej książce & - \\ \hline
/books/<book\_id> & PUT & Zmiana danych książki & body: title, author, format, path, status, user, user\_type \\ \hline
/books/<book\_id> & DELETE & Usuwanie książki & - \\ \hline
/books/<book\_id>/lend/ & GET & Wyświetla informacje o wypożyczeniu & - \\ \hline
/books/<book\_id>/lend/ & POST & Wypożyczanie książki & body: borrower, message, date\_start, date\_stop \\ \hline
/books/<book\_id>/lend/ & PUT & Edycja danych o wypożyczeniu & body: borrower, message, date\_start, date\_stop \\ \hline
/books/<book\_id>/lend/ & DELETE & Usuwa dane wypożyczenie & - \\ \hline
\end{longtable} 


\begin{longtable}{|p{3cm}|p{2cm}|p{3.5cm}|p{6cm}|}
\caption{Akcje związane z wyszukiwaniem} \label{API_2} \\ \hline
\multicolumn{4}{ |c| }{ Wyszukiwanie } \\ 
\multicolumn{4}{ |c| }{ /search } \\ \hline
Endpoint & Request & Opis & Dodatkowo \\ \hline
/search/<query> & GET & Wyszukiwanie & query: opis budowania zapytania w punkcie X.Y \\ \hline
\end{longtable} 


\begin{longtable}{|p{3cm}|p{2cm}|p{3.5cm}|p{6cm}|}
\caption{Akcje związane z instytucjami} \label{API_3} \\ \hline
\multicolumn{4}{ |c| }{ Institutions } \\ 
\multicolumn{4}{ |c| }{ /institutions } \\ \hline
Endpoint & Request & Opis & Dodatkowo \\ \hline
/institutions/ & GET & Lista instytucji & - \\ \hline
/institutions/ & POST & Dodanie instytucji & body: name, url, contact, type, address\_id \\ \hline
/institutions/<login> & GET & Informacje o instytucji & - \\ \hline
/institutions/<login> & PUT & Edycja danych o instytucji &  body: name, url, contact, type, address\_id \\ \hline
/institutions/<login> & DELETE & Usunięcie instytucji & - \\ \hline
\end{longtable} 


\begin{longtable}{|p{3cm}|p{2cm}|p{3.5cm}|p{6cm}|}
\caption{Akcje związane ze znajomymi} \label{API_5} \\ \hline
\multicolumn{4}{ |c| }{ Znajomi } \\ 
\multicolumn{4}{ |c| }{ /friends } \\ \hline
Endpoint & Request & Opis & Dodatkowo \\ \hline
/friends/ & GET & Lista znajomych & - \\ \hline
/friends/ & POST & Dodanie znajomego & body: email \\ \hline
/friends/<login> & GET & Informacje o książkach znajomego & - \\ \hline
/friends/<login> & POST & Wyślij powiadomienie znajomemu & body: type, message \\ \hline
/friends/<login> & DELETE & Usuń znajomego z listy & - \\ \hline
\end{longtable} 


\begin{longtable}{|p{3cm}|p{2cm}|p{3.5cm}|p{6cm}|}
\caption{Akcje związane z wiadomościami} \label{API_6} \\ \hline
\multicolumn{4}{ |c| }{ Wiadomości } \\ 
\multicolumn{4}{ |c| }{ /messages } \\ \hline
Endpoint & Request & Opis & Dodatkowo \\ \hline
/messages/ & GET & Lista wiadomości & - \\ \hline
/messages/<email> & GET & Lista wiadomości od znajomego & -  \\ \hline
/messages/<email> & POST & Wysłanie wiadomości do znajomego & body: message \\ \hline
\end{longtable} 

\chapter{Opis systemu}



\chapter{Spis tabel i obrazów}


\begingroup
\let\clearpage\relax
\listoffigures
\listoftables
\endgroup



\end{document}